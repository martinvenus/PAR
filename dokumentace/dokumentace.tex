\documentclass[12pt]{article}
\usepackage{epsf,epic,eepic,eepicemu}
%\documentstyle[epsf,epic,eepic,eepicemu]{article}
\usepackage[utf8]{inputenc}
\usepackage[czech]{babel}     
\usepackage{ae}
\usepackage{fancyhdr}

\begin{document}
%\oddsidemargin=-5mm \evensidemargin=-5mm \marginparwidth=.08in
%\marginparsep=.01in \marginparpush=5pt \topmargin=-15mm
%\headheight=12pt \headsep=25pt \footheight=12pt \footskip=30pt
%\textheight=25cm \textwidth=17cm \columnsep=2mm \columnseprule=1pt
%\parindent=15pt\parskip=2pt

% ============================ Hlavička dokumentu ==================================================================
\begin{center}
\bf Semestrální projekt MI-PAR 2010/2011\\[5mm]
    Paralelní algoritmus pro řešení problému\\[5mm]
       Jaroslav Líbal\\
       Martin Venuš\\[5mm]
FIT ČVUT\\[2mm]
magisterské studijum\\[2mm]
Kolejní 550/2, 160 00 Praha 6\\[2mm]
29. listopadu 2010
\end{center}

%\pagestyle{empty} %get rid of header/footer for toc page
\newpage
\tableofcontents
\newpage
%\pagestyle{fancy}

% ============================ Definice problému a popis sekvenčního algoritmu ==================================================================
\section{Definice problému a popis sekvenčního algoritmu}
\textbf{Úloha MBG: minimální obarvení grafu}

\subsection{Vstupní data}

$G(V,E)$ = jednoduchý neorietovaný neohodnocený k-regulární graf o n uzlech a m hranách\\
$n$ = přirozené číslo představující počet uzlů grafu G, $n \geq 5$
$k$ = přirozené číslo řádu jednotek představující stupeň uzlu grafu G, $n \geq k \geq 3$; n a k nejsou současně obě liché\\
\\
Doporučení pro generování G:\\
\\
Použijte generátor grafu s volbou typu grafu "-t REG", který vygeneruje souvislý neorientovaný neohodnocený graf.


\subsection{Úkol} 

Nalezněte minimální uzlové obarvení grafu G, či-li zobrazení B: V $\rightarrow$ N takové, že b=$\left|\{B[x]; x\ in\ V\}\right|$ je minimální. Řešení existuje vždy, nemusí být jednoznačné.

\subsection{Výstup algoritmu}

Počet použitých barev a výpis uzlového obarvení $B[1..n]$ grafu G ve formě matice sousednosti (barvy uzlů pište na diagonálu matice).\\

\subsection{Sekvenční algoritmus}

Sekvenční algoritmus je typu BB-DFS s omezenou hloubkou prohledávaného prostoru. Cena, kterou minimalizuje, je počet barev b. Mezistav je dán částečným obarvenim. Mezistav či koncový stav je přípustný, jestliže žádné 2 sousední uzly nejsou obarvené stejnou barvou. Návrat v mezistavu provádíme, pokud nelze žádný neobarvený uzel obarvit, aniž by se porušila podminka uzlového barvení.\\
\\
Triviální horní mez na $b=k+1$ (každý k-regulární graf lze uzlově obarvit $k+1$ barvami).\\
Těsná spodní mez na $b=2$. Pokud je G bipartitní, existuje obarvení dvěma barvami.


\subsection{Paralelní algoritmus}

Paralelní algoritmus je typu G-PBB-DFS-D. 

\subsection{Výhledově odstranit}
Popište problém, který váš program řeší. Jako výchozí použijte text
zadání, který rozšiřte o přesné vymezení všech odchylek, které jste
vůči zadání během implementace provedli (např.  úpravy heuristické
funkce, organizace zásobníku, apod.). Zmiňte i případně i takové
prvky algoritmu, které v zadání nebyly specifikovány, ale které se
ukázaly jako důležité.  Dále popište vstupy a výstupy algoritmu
(formát vstupních a výstupních dat). Uveďte tabulku nameřených časů
sekvenčního algoritmu pro různě velká data.

% ============================ Popis paralelního algoritmu a jeho implementace v MPI ==================================================================
\section{Popis paralelního algoritmu a jeho implementace v MPI}

Popište paralelní algoritmus, opět vyjděte ze zadání a přesně
vymezte odchylky, zvláště u algoritmu pro vyvažování zátěže, hledání
dárce, ci ukončení výpočtu.  Popište a vysvětlete strukturu
celkového paralelního algoritmu na úrovni procesů v MPI a strukturu
kódu jednotlivých procesů. Např. jak je naimplementována smyčka pro
činnost procesů v aktivním stavu i v stavu nečinnosti. Jaké jste
zvolili konstanty a parametry pro škálování algoritmu. Struktura a
sémantika příkazové řádky pro spouštění programu.

% ============================ Naměřené výsledky a vyhodnocení ==================================================================
\section{Naměřené výsledky a vyhodnocení}

Pro měření výsledků byly vygenerovány následující matice:
\begin{table}[ht]
\centering
\begin{tabular}{|l|c|c|c|c|}
\hline \textbf{Matice} & \textbf{Typ matice} & \textbf{Počet vrcholů} & \textbf{Stupeň vrcholu} \\
\hline 
\hline Matice 1 & REG & 42 & 18 \\ 
\hline Matice 2 & REG & 44 & 29 \\ 
\hline Matice 3 & REG & 43 & 12 \\ 
\hline 
\end{tabular}
\caption{Popis jednotlivých vygenerovaných matic.}
\label{matice_popis}	
\end{table}

REG = k-regularni graf

\subsection{Sekvenční řešení}

\begin{table}[ht]
\centering
\begin{tabular}{|l|c|c|c|c|}
\hline \textbf{Typ matice} & \textbf{čas 1 [s]} & \textbf{čas 2 [s]} & \textbf{čas 3 [s]} & \textbf{prům. čas [s]} \\
\hline 
\hline Matice 1 & 364.7 & 364.9 & 365.0 & \textbf{364.86} \\ 
\hline Matice 2 & 652.3 & 651.5 & 651.4 & \textbf{651.7} \\ 
\hline Matice 3 & 1184.7 & 1184.0 & 1184.9 & \textbf{1184.53} \\ 
\hline 
\end{tabular}
\caption{Doba běhu sekvenčního algoritmu pro jednotlivé matice.}
\label{doba_behu_sekvencne}	
\end{table}

\subsection{Paralelní řešení}
\begin{table}[ht]
\centering
\begin{tabular}{|l|c|c|c|c|}
\hline \textbf{Typ matice} & \textbf{Počet procesorů} & \textbf{čas - ethernet [s]} & \textbf{čas - infiniband [s]} \\
\hline 
\hline Matice 1 & 2 & 0.293 & 0.237 \\ 
\hline Matice 2 & 2 & 0.486 & 0.473 \\ 
\hline Matice 3 & 2 & 0.361 & 0.361 \\ 
\hline
\hline Matice 1 & 4 & 0.240 & 0.325 \\ 
\hline Matice 2 & 4 & 0.553 & 0.550 \\ 
\hline Matice 3 & 4 & \textbf{0.419} & 0.362 \\ 
\hline 
\hline Matice 1 & 8 & 0.290 & 0.254 \\ 
\hline Matice 2 & 8 & 0.382 & 0.573 \\ 
\hline Matice 3 & 8 & 0.359 & 0.595 \\ 
\hline 
\hline Matice 1 & 12 & - & - \\ 
\hline Matice 2 & 12 & - & - \\ 
\hline Matice 3 & 12 & - & - \\ 
\hline 
\hline Matice 1 & 16 & - & - \\ 
\hline Matice 2 & 16 & - & - \\ 
\hline Matice 3 & 16 & - & - \\ 
\hline 
\hline Matice 1 & 24 & - & - \\ 
\hline Matice 2 & 24 & - & - \\ 
\hline Matice 3 & 24 & - & - \\ 
\hline 
\end{tabular}
\caption{Doba běhu paralelního algoritmu.}
\label{doba_behu_paralelne}	
\end{table}

\begin{enumerate}
\item Zvolte tři instance problému s takovou velikostí vstupních dat, pro které má
sekvenční algoritmus časovou složitost kolem 5, 10 a 15 minut. Pro
meření čas potřebný na čtení dat z disku a uložení na disk
neuvažujte a zakomentujte ladící tisky, logy, zprávy a výstupy.
\item Měřte paralelní čas při použití $i=2,\cdot,32$ procesorů na sítích Ethernet a InfiniBand.
%\item Pri mereni kazde instance problemu na dany pocet procesoru spoctete pro vas algoritmus dynamicke delby prace celkovy pocet odeslanych zadosti o praci, prumer na 1 procesor a jejich uspesnost.
%\item Mereni pro dany pocet procesoru a instanci problemu provedte 3x a pouzijte prumerne hodnoty.
\item Z naměřených dat sestavte grafy zrychlení $S(n,p)$. Zjistěte, zda a za jakych podmínek
došlo k superlineárnímu zrychlení a pokuste se je zdůvodnit.
\item Vyhodnoďte komunikační složitost dynamického vyvažování zátěže a posuďte
vhodnost vámi implementovaného algoritmu pro hledání dárce a dělení
zásobníku pri řešení vašeho problému. Posuďte efektivnost a
škálovatelnost algoritmu. Popište nedostatky vaší implementace a
navrhněte zlepšení.
\item Empiricky stanovte
granularitu vaší implementace, tj., stupeň paralelismu pro danou
velikost řešeného problému. Stanovte kritéria pro stanovení mezí, za
kterými již není učinné rozkládat výpočet na menší procesy, protože
by komunikační náklady prevážily urychlení paralelním výpočtem.

\end{enumerate}

% ============================ Závěr ==================================================================
\section{Závěr}

Celkové zhodnocení semestrální práce a zkušenosti získaných během
semestru.

% ============================ Definice problému a popis sekvenčního algoritmu ==================================================================
\section{Literatura}

\appendix

\section{Návod pro vkládání grafů a obrázků do Latexu}

Nejjednodušší způsob vytvoření obrázku je použít vektorový grafický
editor (např. xfig nebo jfig), ze kterého lze exportovat buď
\begin{itemize}
\item postscript formáty (ps nebo eps formát) nebo
\item latex formáty (v pořadí prostý latex, latex s macry epic, eepic, eepicemu). Uvedené pořadí odpovídá růstu
komplikovanosti obrázků který formát podporuje (prostá latex macra
umožnují pouze jednoduché, epic makra něco mezi, je třeba
vyzkoušet).

\end{itemize}
Následující příklady platí pro všechny případy.

Obrázek v postscriptu, vycentrovaný a na celou šířku stránky, s
popisem a číslem. Všimnete si, jak řídit velikost obrazku.

Obrázek pouze vložený mezi řádky textu, bez popisu a číslování.\\

Latexovské obrázky maji přípony *.latex, *.epic, *.eepic, a
*.eepicemu, respective.
\begin{figure}[ht]
\begin{center}
\end{center}
\caption{Popis vašeho obrázku} \label{l1}
\end{figure}

bez čísla a popisu.

Takhle jednoduše můžete poskládat obrázky vedle sebe.

Řídit velikost latexovskych obrázků lze příkazem
\begin{verbatim}
\setlength{\unitlength}{0.1mm}
\end{verbatim}
které mění měřítko rastru obrázku, Tyto příkazy je ale současně
nutné vyhodit ze souboru, který xfig vygeneroval.

Pro vytváření grafu lze použít program gnuplot, který umí generovat
postscriptovy soubor, ktery vložíte do Latexu výše uvedeným
způsobem.

\end{document}
